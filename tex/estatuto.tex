\documentclass[12pt]{article}
\usepackage[utf8]{inputenc}
\usepackage[T1]{fontenc}
\usepackage[brazil]{babel}
\usepackage{indentfirst}
\usepackage[a4paper]{geometry}
\usepackage{graphicx}
\usepackage{enumitem}
\usepackage{gensymb}
\title{Estatuto do Centro Acadêmico de Ciência da Computação de Sorocaba}
\author{Centro Acadêmico Pata do Bisão}

\begin{document}
\maketitle 

%%% OBJETIVO DO ESTATUTO %%%
\section{Capítulo I - Objetivo do Projeto}
    \S 1$^{\circ}$ -- O Estatuto tem como principal objetivo determinar metas e
    regras, para organização e harmonia, dentro do Centro Acadêmico. Este foi
    oficializado por meio de votação dos membros do curso de graduação de
    Ciência da Computação da UFSCar Sorocaba com voto aberto no dia 27 de
    setembro de 2013.

%%% OBJETIVO DO CA %%%
\section{Capítulo II - Objetivos, Princípios e Finalidades}
\begin{description}
\item[Artigo 1$^\circ$]{O Centro Acadêmico, assim como seus membros
devidamente matriculados no curso de Ciência da Computação, tem como principais
objetivos, princípios e finalidades os itens a seguir:}
        \begin{enumerate}[label=(\alph*)]
        \item{Representar os estudantes de graduação da Universidade Federal de São
        Carlos, campus Sorocaba, do curso de Ciência da Computação, no todo ou em
        parte, judicial ou extra-judicialmente;}
        \item Defender os interesses gerais dos estudantes;
        \item{Promover e incentivar todas as formas de organização dos estudantes
        reconhecidas pelo Congresso de Estudantes da UFSCar;}
        \item{Cooperar com os estudantes secundaristas e com suas entidades
        representativas;}
        \item{Incentivar as relações amistosas entre as organizações estudantis;}
        \item{Defender o ensino público, gratuito, de qualidade e para todos
        referenciados socialmente;}
        \item Lutar pelo livre acesso à educação pública;
        \item{Defender a democracia e as liberdades fundamentais do homem e da
        mulher;}
        \item{Divulgar e discutir as bandeiras e resoluções do Congresso, dentro
        dos fóruns do movimento estudantil da UFSCar (CAs, CCAs, Assembléias) e da
        UNE, da UEE-SP e para a sociedade.}
        \item{Difusão e fomento de atividades culturais, artísticas e políticas do
        campus entre os estudantes e a sociedade;}
        \item{Desenvolver, divulgar e participar de atividades e eventos que
        enriqueçam a vivência acadêmica de todos os alunos do curso e/ou campus.}
    \end{enumerate}
\item[Artigo 2$^\circ$]{O Centro Acadêmico poderá firmar convênios, intercâmbios e
iniciativas conjuntas com organizações e entidades públicas ou privadas,
nacionais ou estrangeiras, bem como firmar parcerias com estas mesmas
entidades.}
\end{description}

%%% ORGANIZAÇÃO DA ENTIDADE %%%
\section{Capítulo III -- Da Organização da Entidade}
    \begin{enumerate}[label=\S\,\arabic*$^\circ$]
        \item{O Centro Acadêmico tem sua estrutura dividida em três diretorias
            e grupos de trabalho. Cada diretor tem responsabilidades referentes
            apenas às suas áreas podendo ou não participar dos grupos de
            trabalho. Além disso, cada grupo de trabalho tem um líder que é
            responsável no andamento dos projetos.}
        \item{Nas decisões de projetos ou discussões, o voto é unitário e
            válido na mesma intensidade para qualquer cargo ocupado dentro do
            Centro Acadêmico. Em caso de empate, a votação será decidida pela
            presidência.}
        \item{Independentemente de cargos dos membros do Centro Acadêmico,
            todos sem exceção devem contribuir para o desenvolvimento das
            atividades feitas por outra área ou divisão, caso solicitado o
            auxílio. A organização e separação de responsabilidades individuais
            serão feitas durante as reuniões do Centro Acadêmico, dividindo os
            afazeres e responsabilidades das determinadas atividades.}
        \item{O tempo de gestão dentro de qualquer cargo do Centro Acadêmico é de um ano, a menos do membro comum.}
        \item{As reuniões devem ser periódicas e abertas para todos os
            interessados, sendo que o quórum mínimo para que qualquer decisão
            seja efetivamente tomada é de 50\% dos integrantes da chapa e
            membros ativos}
        \item{As reuniões devem ser sempre públicas e a divulgação feita com 24
            horas de antecedência, podendo ser feita nos espaços da
            universidade ou na página do Facebook do Centro Acadêmico;}
        \item{Caso haja desistência de algum diretor, haverá uma votação
            interna para decidir qual dos membros passará a exercer o cargo
            vago. Se houver contestação por parte dos estudantes, uma
            assembleia será organizada para que a decisão seja votada. }
    \end{enumerate}

\subsection*{Estrutura da Entidade}
Serão atribuídos os seguintes cargos aos membros:
\begin{itemize}
    \item Presidente e Vice-Presidente
        \subitem Descrição: Têm a responsabilidade de coordenar a
        análise dos projetos propostos por qualquer membro do Centro Acadêmico
        e de qualquer estudante matriculado no curso de Ciência da Computação,
        UFSCar Campus Sorocaba. Discutindo analiticamente e estatisticamente a
        possibilidade de desenvolvimento do projeto.
        \subitem O presidente de cada gestão ainda tem como responsabilidade a chave do
        armário do Centro Acadêmico e da sala na qual o armário estiver
        localizado.
        \subitem Vagas: 2
    \item Tesoureiro
        \subitem Tem a responsabilidade de gerenciar e desenvolver, qualquer
        atividade na área de finanças do Centro Acadêmico. Para manter um bom
        andamento e confiabilidade de operações, devem-se ter duas pessoas
        nesta diretoria, de modo que uma possa vistoriar o trabalho da outra.
        \subitem Vagas: 2

    \item Secretário
        \subitem Tem a responsabilidade de desenvolver e analisar toda a parte
        burocrática do Centro Acadêmico e registrar as pautas e atas das
        reuniões.
        \subitem Vagas: 1

    \item Diretor de Comunicação
        \subitem Tem a responsabilidade de ser o meio de comunicação, de
        qualquer espécie, entre os estudantes e os docentes/coordenação e entre
        a universidade e qualquer organização ou pessoa física sem vínculo à
        universidade. Também é responsável pelo desenvolvimento de atividades
        para ampliar os meios de comunicação e de divulgação do curso e da
        universidade.
        \subitem Vagas: 1

    \item Diretor Sociocultural
        \subitem Tem a responsabilidade de promover eventos sociais e culturas
        do Centro Acadêmico de qualquer espécie dentro da universidade.
        \subitem Vagas: 1

    \item Membro
        \subitem É considerado membro do Centro Acadêmico com poder de voto
        qualquer aluno do curso de Ciência da Computação da UFSCar campus
        Sorocaba que manifeste interesse e participe de pelo menos duas
        reuniões e um projeto do Centro Acadêmico ou aquele admitido membro por
        votação pela atual chapa.
        \subitem Vagas: Ilimitadas

    \item Grupos de Trabalho
        \subitem Um grupo de trabalho é criado a partir da necessidade de
        membros para o desenvolvimento de um projeto, sendo inserido na ata
        referente à reunião em que foi criado. Cada Grupo conta com um
        líder, que é o responsável por deixar todos os outros membros do Centro
        Acadêmico a par do andamento do projeto.
        \subitem  Quantidade de Grupos: Proporcional à quantidade de projetos
\end{itemize}


%%% PROCESSO ELEITORAL %%%
\section{Capítulo IV -- Do Processo Eleitoral}
\begin{description}
    \item[Artigo 3$^\circ$] A eleição entre chapas concorrentes ao Centro
        Acadêmico deverá acontecer a cada ano, visto que a vigência dos cargos
        e da chapa ganhadora é de um ano.
\end{description}
\begin{enumerate}[label= \S\,\arabic*$^\circ$]
    \item A eleição deverá acontecer sempre no final do primeiro semestre de
        cada ano, de modo que, no semestre seguinte, a chapa vencedora já possa
        exercer suas funções.
    \item É de responsabilidade da gestão atual garantir a existência de uma
        comissão eleitoral para que a próxima eleição possa acontecer
    \item A comissão eleitoral deve sempre ser composta por no mínimo dois
        membros, sendo estes não-relacionado às chapas e responsáveis pela
        \begin{enumerate}[label=\alph*)]
            \item decisão das normas e a manutenção do processo;
            \item contagem dos votos e divulgação de resultado;
        \end{enumerate}
    \item Toda a divulgação da chapa e suas propostas devem ser feitas pelas
        mesmas.
    \item No caso de apenas uma chapa inscrita, a votação ocorrerá pela
        aceitação ou não da mesma, sendo que o método utilizado para a
        validação dos votos é o de maioria simples. Caso a chapa seja recusada,
        a chapa anterior se mantém como provisória e deverá ocorrer outra
        eleição assim que outro grupo se interessar pela entidade.
    \item Passado o processo, os novos integrantes do C.A. têm por
        responsabilidade notificar da maneira que considerar mais viável todas as outras entidades estudantis, assim
        como docentes e coordenação do curso.
\end{enumerate}

%%% HIERARQUIA DAS ENTIDADES %%%
\section{Capítulo V -- Da Hierarquia das Entidades}
\begin{description}
    \item[Artigo 4$^\circ$] São instâncias deliberativas do Centro Acadêmico
        UFSCar, com poder decrescente de deliberação, nesta ordem:
        \begin{enumerate}[label=\alph*)]
            \item Assembleia dos Estudantes
                \subitem É de responsabilidade do Centro Acadêmico convocar
                assembleias entre os estudantes sempre que os membros acharem
                necessário ou quando, de maneira organizada, os alunos
                trouxerem pauta pertinente para discussão e posterior votação.
                Quando trazida por alunos, as pautas devem ser aprovadas por
                pelo menos 20\% de todo o curso por meio de assinatura, caso
                contrário não será necessária a convocação da assembleia.
            \item Conselho dos integrantes dos Centros Acadêmicos;
            \item Diretoria Geral;
        \end{enumerate}
    
\end{description}

%%% RESPONSABILIDADES E PUNIÇÕES %%%
\section{Capítulo VI -- Responsabilidades e Punições}
\begin{enumerate}[label=\S\,\arabic*$^\circ$]
    \item Todo membro do Centro Acadêmico tem a responsabilidade de lutar para
        o bem-estar dos estudantes e da Universidade com seriedade, empenho e
        firmeza, cumprindo com os deveres do cargo ocupado.
    \item Se existir evidências que algum membro está agindo de má-fé e/ou não
        se empenhando dentro do Centro Acadêmico, uma Assembleia entre os
        membros deverá ser solicitada e, então, deverão ser apresentadas as
        provas/evidências de tal conduta. A conclusão dessa Assembleia será uma
        votação aberta para o afastamento ou não do membro de seu cargo. O
        afastamento não impede o membro de participar das reuniões, apenas
        retira o seu direito de voto até a próxima gestão ou até nova
        Assembleia.
    \item Ao fim de cada festão, o setor financeiro do Centro Acadêmico deve
        disponibilizar para a próxima gestão a movimentação do caixa.
    \item Todo e qualquer problema relacionado às finanças do Centro Acadêmico
        é de responsabilidade das pessoas que compõem a tesouraria.
    \item Se um membro solicitar uma quantia do caixa do Centro Acadêmico é
        transferia a ele a responsabilidade referete àquela quantia, tendo como
        garantia evidências.
        \subitem São consideradas evidências: recibos, notas, atas de reunião e
        comprovantes.
    \item É responsabilidade de um membro participar ativamente das reuniões e
        estar, em ata, em pelo menos um projeto. A participação em projeto é
        avaliada pelo líder do Grupo de Trabalho.
\end{enumerate}

\begin{description}
    \item[Artigo 7$^\circ$] O afastamento do membro de um cargo será feito sem
        contestação nas seguintes situações:
        \begin{enumerate}[label=arabic*.]
            \item Falta de respeito com qualquer estudante, docente ou qualquer
                membro dentro das atividades do Centro Acadêmico, tais como
                agressões físicas ou verbais de qualquer espécie.
            \item Ação mal-intencionada dentro das suas funções ou em outras
                dentro do Centro Acadêmico, tais como malícias que levem ao
                mal-estar dos estudantes, desvio de verba
                ou interferência de má-fé em outro setor.
            \item Duas faltas consecutivas, sem justificativa, nas reuniões do
                Centro Acadêmico farão com que o membto seja considerado
                ausente e desinteressado.
        \end{enumerate}
\end{description}
\end{document}

